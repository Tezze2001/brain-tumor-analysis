\chapter{Bayes} \label{chp:bayes}
La precedente fase di analisi ha permesso di acquisire informazioni utili sulla
struttura del dataset e di conseguenza permettere la selezione di un modello
adatto a svolgere il compito di classificazione.

In questo capitolo verranno presentati tutti i risultati ottenuti dall'apprendimento
e dalle valutazioni effettuate sul modello Gaussian Naive Bayes. Da notare che 
si sta utilizzando Gaussian Naive Bayes pur sapendo che non tutte le features 
derivano da una distribuzione normale, siamo consci del fatto che non si stanno 
rispettando le assunzioni del modello.

\section{Preparazione dei dati}
La prima operazione svolta sui dati è stata la suddivisione del dataset in
training set e test set. Il training set è stato utilizzato per addestrare la
rete neurale, mentre il test set è stato utilizzato per valutare le prestazioni
della rete neurale. La suddivisione del dataset è stata effettuata in modo tale
che il training set contenesse il $80\%$ dei dati, mentre il test set contenesse
il $20\%$ dei dati.

La suddivisione dei dati è stata effettuata in modo da mantenere la stessa 
percentuale di dati positivi e negativi in entrambi i set. Questa operazione è
stata effettuata per evitare che la rete neurale sia addestrata su un dataset
sbilanciato.

\section{Addestramento di Gaussian Naive Bayes}
Dal momento che il modello non ha degli iperparametri da stimare allora non è 
stato effettuato il processo di tuning degli iperparametri come è stato effettuato
per la rete neurale. Di conseguenza è stato allenato direttamente il modello sul
training set e successivamente è stata effettuata la sua valutazione calcolando 
le metriche di valutazione.

\section{Risultati}
Il modello addestrato in precedenza è stato valutato sui dati che compongono il
test set. In particolare sono state valutate le seguenti metriche: accuratezza,
precisione, richiamo e F1 score. Oltre al calcolo di queste metriche, si è 
deciso di realizzare la curva ROC per il modello e di rappresentare la matrice
di confusione.

Prima di presentare i risultati ottenuti, è necessario specificare che essendo 
il dataset riferito a un ambito medico, si è deciso di aggiustare il valore di
threshold per la predizione del modello. In particolare, il valore di threshold
è stato impostato a $0.3$, in modo tale da ridurre il numero di falsi negativi.

Fatta questa precisazione, si può procedere con la presentazione dei risultati
ottenuti. In particolare, nella tabella \ref{tab:risultatiReteNeurale} sono
presentati i risultati ottenuti dal modello addestrato.

\begin{table}[ht]
    \centering
    \begin{tabular}{|c|c|}
        \hline
        \textbf{Metrica} & \textbf{Valore} \\
        \hline
        Accuratezza & ?? \\
        \hline
        Precisione & ?? \\
        \hline
        Richiamo & ?? \\
        \hline
        F1 score & ?? \\
        \hline
    \end{tabular}
    \caption{Risultati ottenuti dal modello addestrato}
    \label{tab:risultatiReteNeurale}
\end{table}

I risultati ottenuti sono giustificati dal fatto che le due classi sono linearmente
separabili.

% Lo facciamo alla fine? \section{Confronto con il percettrone}