\chapter{Introduzione}

Questo è un progetto per l'esame di Machine Learning del primo anno del corso 
di laurea magistrale in informatica dell'Università degli Studi di Milano-Bicocca.

L'intero progetto si basa sul riconoscimento della presenza di un tumore al cervello
data l'immagine di una risonanza magnetica. Il dataset scelto per questo progetto
è scaricabile dal seguente \href{https://www.kaggle.com/datasets/jakeshbohaju/brain-tumor/data}{link}.

Per il riconoscimento del tumore sono stati allenati i seguenti modelli di machine
learning:
\begin{itemize}
    \item \textbf{SVM}: è stato scelto questo modello vista la buona capacità
    teorica nel generalizzare.
    \item \textbf{Naive Bayes Gaussiano}: è stato scelto questo modello dal momento
    che è l'unico ad essere probabilistico.
     % TODO: anche la rete neurale e SVM possono essere probabilistica
    \item \textbf{Rete neurale}: è stato scelto questo modello per confrontare 
    i primi due con una soluzione neurale.
\end{itemize}

La relazione è stata suddivisa nei seguenti capitoli:
\begin{itemize}
    \item \textbf{Introduzione}: descrizione del dominio e presentazione dei modelli che 
    verranno presi in considerazione per questo progetto.
    \item \textbf{Dataset}: descrizione di come è stato costruito il dataset a partire
    dalle immagini, ovvero come sono state ricavate le features, e analisi esplorativa.
    \item \textbf{Rete neurale}: descrizione e analisi delle performance della rete.
    \item \textbf{SVM}: descrizione e analisi delle performance delle SVM.
    \item \textbf{Naive Bayes Gaussiano}: descrizione e analisi delle performance 
    per Naive Bayes Gaussian.
    \item \textbf{Analisi dei risultati}: analisi comparata dei risultati tra i 
    tre modelli considerati.
    \item \textbf{Conclusioni}: conclusioni sull'elaborato.
\end{itemize}