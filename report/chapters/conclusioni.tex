\chapter{Conclusioni}
L'intero progetto si è basato sul riconoscimento della presenza del tumore del
cervello a partire da immagini in bianco e nero prodotte dalla risonanza
magnetica dei pazienti.

Prima di tutto, è stato chiarito il processo di estrazione delle caratteristiche
dalle immagini della risonanza magnetica. Ciò ha consentito di comprendere il
significato delle caratteristiche calcolate, un passaggio fondamentale per
analizzare in dettaglio tutte le analisi esplorative condotte sul dataset.

Esaminata la composizione del dataset, è stata condotta un'analisi esplorativa
durante la quale sono state applicate varie trasformazioni per eliminare
eventuali valori nulli o costanti. Successivamente, si è proseguito valutando
l'equilibrio delle classi nel dataset e analizzando le distribuzioni delle
feature attraverso una rappresentazione grafica. Da queste indagini è emerso che
le classi presenti nel dataset sono bilanciate e che alcuni attributi non presentano
una distribuzione normale.

In seguito all'analisi delle distribuzioni, sono stati generati box plot per
ciascuna caratteristica, distinguendo tra le due classi del dataset. Questo
approccio ha consentito di identificare eventuali attributi costanti e
caratteristiche altamente discriminanti.

Durante l'analisi esplorativa, sono stati condotti studi sulle correlazioni tra
le caratteristiche al fine di identificare possibili relazioni tra gli attributi.
Da questa fase è emerso che diverse correlazioni sono presenti tra le
caratteristiche che misurano la distribuzione dei livelli di grigio e quelle
legate alla valutazione del contrasto e dell'omogeneità delle texture.

Successivamente, al termine dell'analisi esplorativa, è stata necessaria una
riduzione della dimensionalità prima di fornire i dati agli algoritmi di machine
learning. Notando le correlazioni tra gli attributi, si è deciso di non limitarsi
alla sola riduzione dimensionale con l'analisi delle componenti principali (PCA),
ma di considerare anche la rimozione delle correlazioni. Di conseguenza, sono
stati creati due dataset distinti: \texttt{dataset\_corr} e \texttt{dataset\_pca},
al fine di confrontare non solo i modelli, ma anche i due metodi di riduzione
dimensionale.

Dopo la creazione dei due dataset ridotti, sono stati valutati i modelli
selezionati su entrambi. La valutazione è stata suddivisa in due fasi:
\begin{itemize}
    \item La prima consisteva nella divisione di ciascun dataset in train e test
          per l'allenamento e la valutazione dei modelli.
    \item La seconda fase ha coinvolto una cross-validation per calcolare gli
          intervalli di confidenza delle metriche.
\end{itemize}
Per i modelli che richiedevano l'ottimizzazione degli iperparametri, questa è
stata eseguita tramite cross-validation sul train set della prima valutazione,
utilizzando gli stessi iperparametri per la seconda valutazione.

In merito ai risultati ottenuti dalla valutazione dei modelli, emerge che tutti
e tre i modelli sono efficaci classificatori per il problema, con valori
superiori al $90\%$ per ogni metrica di valutazione. Un'analisi più approfondita
mostra che la metodologia di riduzione della dimensionalità ha un impatto
limitato sui risultati, con metriche e intervalli molto simili sia nella prima
che nella seconda valutazione. Si osserva però un miglioramento dal punto di
vista computazionale, con tempi di addestramento inferiori per i modelli addestrati.

Tra i modelli, la rete neurale e il SVM si distinguono per le loro prestazioni
superiori, mentre per quanto riguarda Gaussian Naive Bayes non si riescono a
raggiunge gli stessi risultati, specialmente per la Recall, la quale è la metrica
più importante nel contesto del problema di classificazione dei tumori. Questo
perché è preferibile avere un falso positivo piuttosto che un falso negativo,
in quanto un falso negativo potrebbe portare a non diagnosticare la presenza di
un tumore.

In conclusione, tutti e tre i modelli si sono dimostrati validi per la
classificazione dei tumori. Tuttavia, i risultati indicano che il modello SVM
eccelle in termini di precisione e efficienza temporale rispetto alla rete
neurale, considerando anche i tempi di addestramento e di ottimizzazione degli
iperparametri, che si sono rivelati inferiori per SVM.