\chapter{Dataset}
Il dataset è stato a partire da un set di $3762$ immagini ottenute dalla risonanza
magnetica del cervello di $3762$ persone, ettichettato manualmente da professionisti
del settore nelle rispettive classi:
\begin{itemize}
    \item \textbf{presenza del tumore}: $T = 1$
    \item \textbf{assenza del tumore}: $T = 0$
\end{itemize} 
Il valore della label cade sotto al colonna \textit{Class}.

Le features del dataset sono state ottenute calcolando i \textbf{momenti Hu}  sulle 
immagini della risonanza magnetica. I momenti Hu catturano le informazioni di base
sull'immagine come l'area dell'oggetto, il centroide, l'orientazione e altre proprietà.

Le feature sulle immagini si dividono in base a 2 gruppi\cite{explanation-features}:
\begin{itemize}
    \item \textbf{First Order Features}: forniscono informazioni legate alla
    distribuzione dei livelli di grigio dell'immagine. Queste features corrispondono
    alle statistiche descrittive calcolate sui valori di ciascun pixel dell'imamgine:
    \begin{itemize}
        \item \textbf{media}
        \item \textbf{varianza}
        \item \textbf{deviazione standard}
        \item \textbf{indice di asimmetria}
        \item \textbf{indice di kurtosis}
    \end{itemize}
    \item \textbf{Second Order Features}: forniscono informazioni a livello di 
    composizione della texture dell'immagine. 
    \begin{itemize}
        \item \textbf{contrast}: misura la variazione locale dei livelli di grigio 
        dei pixel. Maggiore sarà il valore allora maggiore sarà il contrasto dell'immagine.
        \item \textbf{energy}: misura l'etereogenità o la variazione dell'intensità
        dell'immagine. Più piccolo è il valore allora meno variazioni di intesità e
        più omogenea sarà la texture, al contrario, più la texture è irregolare 
        allora maggiore sarà il valore.
        \item \textbf{ASM}: misura come sono distribuiti uniformemente i livelli
        di grigio nell'immagine. Più uniforfe e distribuiti sono i livelli di grigio
        allora minore sarà il valore dell'indice.
        \item \textbf{entropy}: misura la randomicità dei livelli di grigio, quindi
        più piccolo è il valore, più la texture sarà uniforme.
        \item \textbf{homogeneous}: misura secondaria del contrasto. Più alto sarà
        l'indice allora minore sarà il contrasto dell'immagine.
        \item \textbf{dissimilarity}: misura quanto spesso diffenti combinazioni dei valori
        di intensità dei pixel occorrono nell'immagine. Un valore alto dell'indice 
        indica che l'immagine ha una magior variazioni delle intesità dei pixel vicini,
        quindi più complessa sarà la texture.
        \item \textbf{correlation}: misura la correlazione tra pixel nelle due diverse
        direzioni.
        \item \textbf{coarseness}: misura quanto l'immagine è composta da regioni 
        di intensità omogenea, ovvero quanto è granulare o fine la texture.
    \end{itemize}
\end{itemize}



\section{Analisi descrittiva}
Il dataset è formato da un totale composto da un totale $3762$ esempi, ciascuno
descritto da $15$ attributi: $13$ legati alle feature (presentate precedentemente) 
calcolate sull'immagine associata, mentre l'attributo \textit{Image} specifica l'id
dell'immagine a cui si riferisce l'esempio, infine, l'attributo \textit{Class}
specifica l'etichetta dell'esempio.

Quando è stato importato il dataset nel dataframe, è stato opportuno controllare che il tipo
inferito in automatico sulla colonna fosse corretto, di conseguenza è stata convertita
tutta la colonna associata alla label in \textit{categorico}, infine, tutti gli altri attributi sono
stati mantenuti di tipo \textit{float}, essendo valori continui. Per altro si evidenza
l'assenza di valori nulli, evitando di doversi operare per rimuoverli dal dataset.

Successivamente sono state calcolate le statistiche descrittive del dataframe. Ciò
ha permesso innanzitutto di assumere che la distribuzione delle singole feature
come una distribuzione continua normale, per di più si evince che i dati non sono
standardizzati, dal momento che la media e la deviazione standard differiscono rispettivamente
da $0$ e $1$. Quindi fondamentale sarà l'operazione di standardizzazione delle feature
mediante l'equazione \ref{eq:std-features}.

\begin{equation}
    F_{\mu, \sigma} = \frac{F - \mu}{\sigma}
    \label{eq:std-features}
\end{equation}

Dalle statistiche descrittive si può notare come la feature di \textit{Coarseness}
assume valori molto bassi, quindi è stato pensato di convertire questa feature ad una
scala logaritmica, permettendo di aumentare la significatività della feature. In 
ogni caso si può notare che i valori della feature hanno comunque una deviazione standard
pari a circa $0.02$, valore non particolarmente significativo, quindi ci permette
di escludere questa feature perché sicuramente non sarà la feature più discriminante.

Una delle operazioni preliminari di analisi del dataset è controllare il bilanciamento
delle etichette di ciascun esempio, in questo modo è possibile valutare se il
dataset in esame è buono per essere utilizzato per l'apprendimento supervisionato.
In questo caso disegnando lo scatter plot delle label si può notare che le classi
sono bilanciate, infatti, circa il $55\%$ degli esempi sono negativi (assenza del tumore),
contro circa il $45\%$ degli esempi sono positivi (presenza del tumore).

Successivamente risulta importante analizzare se le distribuzioni si avvicinano 
ad una normale standard, di conseguenza sono stati disegnati i barplot delle frequenze
di ciascuna feature. Dai grafici si evince che tutte le feature eccetto \textit{Energy} e
\textit{Homogeneity} sono distribuite normalmente. Di queste si può notare come  
solo la feature di \textit{Standard deviation} è simmentrica, mentre \textit{Mean},
\textit{variance}, \textit{Entropy}, \textit{Skewness}, \textit{Contrast}, \textit{ASM},
\textit{Dissimilarity} e \textit{Coarseness}. Al contrario, si ha una simmetria
negativa solo per la feature di \textit{Contrast}.

\subsection{Analisi delle correlazioni}
Il passo successivo è calcolare le correlazioni tra le feature, in modo tale da
considerare tra le feature correlate a 2 a 2 solo una. Dalla matrice si evince
come la \textit{Coarseness} è scorrelata rispetto alle altre feature, questo è
giustificato dal fatto che è l'unica che ha una deviazione standard tendente a 0.
Al contempo dalla matrice si evince che diverse tendenze di forte correlazione tra
diverse coppie di feature. 

Si potrebbe, innanzitutto, cominciare dalla forte correlazione positiva
tra \textbf{media} e \textbf{variabilità}, più precisamente tra le feature \textit{Mean},
\textit{variance} e \textit{standard deviation}. La correlazione tra varianza è 
deviazione standard è banale ($SD[P] = \sqrt{VAR[P]}$), invece, la correlazione tra media 
e variabilità può essere indotta da come sono composte le immagini. Più precisamente 
analizzando le immagini prodotte dalle risonanze magnetiche, si evince che, essendo in 
bianco e nero, se la media tende a $1$ (colore bianco) allora deve aumentare la 
variabilità, perché ci sono diversi pixel bianchi allora le transizioni dal nero assoluto
al bianco assoluto neccessitano di regioni di pixel maggiore rispetto ad una 
transizione tra nero assoluto e grigio ($0.5$). 

Una seconda forte correlazione positiva è 
tra le feature che misurano l'\textbf{uniformità dei livelli di grigio} dei pixel,
più precisamente tra le feature \textit{Entropy}, \textit{ASM}, \textit{Homogeneity} 
ed \textit{Energy}. Queste feature quantificano delle informazioni legate alla
texture dell'immagine, quindi la forte correlazione positiva può essere facilmente
spiegata analizzando le texture delle immagini su cui vengono calcolate. Più precisamente
se si ha un valore molto alto della feature \textit{Entropy}, significa che la
texture non è uniforme, ovvero si hanno strutture complesse e irregolari, quindi
meno uniforme sarà la distribuzione dei livelli di grigio, aumentando l'indice
di \textit{ASM}, comportando di conseguenza un aumento delle variazioni di intensità
dei livelli di grigio, aumentando di conseguenza anche l'indice di \textit{Energy},
infine, (\textbf{cosa c'entra Homogeneity?}). 

Al tempo stesso la matrice di correlazione evidenza una forte correlazione positiva
tra gli indici che misurano la \textbf{morfologia della distribuzione}, ovvero 
le feature di \textit{Skewness} e \textit{Kurtosis}. Questa dipendenza implica il 
fatto che più la distribuzione è leptokurtica (Kurtosis grande), ovvero la frequenza
dei livelli di grigio dei pixel si concentrano interamente vicino alla media/mediana/
moda, allora più grande sarà la Skewness, ovvero maggiore sarà la tendenza ad avere
frequenze di livelli di grigio più vicino al bianco (coda di destra più altra rispetto
alla coda di sinistra).

La matrice della correlazione evidenzia anche una correlazione positiva tra le 
feature di \textit{Contrast} e \textit{Dissimilarity}, ovvero maggiore sarà il 
contrasto e maggiore sarà la complessità della texture. 

In aggiunta dalla matrice si evidenza che le features di \textit{Dissimilarity}
e \textit{Homogeneity} sono correlate negativo, ovvero maggiore sarà il contrasto
allora minore è la complessità della texture.