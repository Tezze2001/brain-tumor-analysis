\chapter{Dataset}
Il dataset è stato a partire da un set di $3762$ immagini ottenute dalla risonanza
magnetica del cervello di $3762$ persone, ettichettato manualmente da professionisti
del settore nelle rispettive classi:
\begin{itemize}
    \item \textbf{presenza del tumore}: $T = 1$
    \item \textbf{assenza del tumore}: $T = 0$
\end{itemize} 
Il valore della label cade sotto al colonna \textit{Class}.

Le features del dataset sono state ottenute calcolando i \textbf{momenti Hu}  sulle 
immagini della risonanza magnetica. I momenti Hu catturano le informazioni di base
sull'immagine come l'area dell'oggetto, il centroide, l'orientazione e altre proprietà.

Le feature sulle immagini si dividono in base a 2 gruppi\cite{explanation-features}:
\begin{itemize}
    \item \textbf{First Order Features}: forniscono informazioni legate alla
    distribuzione dei livelli di grigio dell'immagine. Queste features corrispondono
    alle statistiche descrittive calcolate sui valori di ciascun pixel dell'imamgine:
    \begin{itemize}
        \item \textbf{media}
        \item \textbf{varianza}
        \item \textbf{deviazione standard}
        \item \textbf{indice di asimmetria}
        \item \textbf{indice di kurtosis}
    \end{itemize}
    \item \textbf{Second Order Features}: forniscono informazioni a livello di 
    composizione della texture dell'immagine. 
    \begin{itemize}
        \item \textbf{contrast}
        \item \textbf{energy}
        \item \textbf{asm}
        \item \textbf{entropy}
        \item \textbf{homogeneous}
        \item \textbf{dissimilarity}
        \item \textbf{correlation}
        \item \textbf{coarseness}
    \end{itemize}
\end{itemize}
